\documentclass[12pt]{article}
\usepackage[a4paper]{geometry}
\usepackage[english]{babel}   % Will manage the 90% hyphenation for us 
%\usepackage[parfill]{parskip} % Activate to begin paragraphs width empty line instead of indent
\usepackage{graphicx}
\usepackage{amssymb,amsmath}
\usepackage{setspace}
\usepackage{url}
\usepackage{natbib}
\usepackage{pdfpages}
\usepackage{longtable}
\usepackage{booktabs}
\usepackage[linktoc=all,colorlinks=true,citecolor=darkgray,filecolor=black,linkcolor=blue,urlcolor=purple]{hyperref}

%%%% JN: For verbatim blocks, combine features from these packages
\usepackage{float}
\usepackage{fancyvrb}
\usepackage{etoolbox}
\makeatletter
\preto{\@verbatim}{\topsep=0pt \partopsep=0pt }
\makeatother
\RecustomVerbatimEnvironment
    {BVerbatim}{BVerbatim}
    {baselinestretch=0.8,fontsize=\footnotesize}
\RecustomVerbatimEnvironment
    {Verbatim}{Verbatim}
    {baselinestretch=0.8,fontsize=\small}

% The following specifies a line spacing of 1.2 times normal
\renewcommand{\baselinestretch}{1.2}

% Short-hand font settings used in this document:
%%%% JN: Currently, tttext is used instead of ttt. prout and term are not used
\usepackage[T1]{fontenc}
\usepackage{lmodern}
\renewcommand\ttfamily{\usefont{T1}{lmtt}{m}{n}}
\newcommand{\term}[1]{{\bfseries\slshape #1}}
\newcommand{\ttt}[1]{\texttt{#1}}
\newcommand{\prout}[1]{\ttt{#1}}
\newcommand{\tb}[1]{\ttt{\textbf{#1}}}

%%%% JN: Draft version, July 2020
%%%% JN: I (JN) have started to review the text, re-run the commands, and added occasional comments.
%%%% JN: Note that I use automatic text wrapping for paragraphs. In vim:
%%%% JN  ':set textwidth=99' followed by ':set colorcolumn=+1'
%%%% JN: NOTES:
%%%% JN: Since we are not yet ready with the text, placement of tables (figures) are
%%%% JN: not a priority. This will have to be set at a later stage.

\begin{document}

\title{DiemthylBayes Tutorial}

\date{\large Draft version, 3/2025}

\author{Greg Ellison}

\maketitle

\tableofcontents

\section{Introduction}\label{intro}

This short document describes the use of DimethylBayes, an extension of MrBayes
with features to support phylogenetic inference of DNA methylation data. All the 
features of MrBayes are available, so the online reference for MrBayes 
should be referred to for detailed instructions for use.
This document is a supplement that describes the 
new features available for DNA methylation data. 

\section{Acquiring and Installing DimethylBayes}

DimethylBayes is available from https://github.com/gmellison/DimethylBayes.
The installation process is identical to MrBayes. 

\section{A Simple Analysis}\label{tutorialSimple}

\subsection{Dimethyl Data in MrBayes}

The Dimethyl datatype extends MrBayes by allowing methylome alignments 
in addition to the datatypes described in the MrBayes manual. The 
dimethyl datatype allows the characters \{0, 1, 2\} to represent the 
number of methyl group attachments at a nucleotide site. Missing or 
unknown characters are also allowed. There is an example nexus file 
called \texttt{dimethyl.nex} in the examples folder, which contains the 
following \texttt{DATA} block:

\begin{figure}[h]
\centering
\begin{Verbatim}
begin data;
dimensions ntax=20 nchar=150;
format datatype=dimethyl interleave=yes gap=- missing=?;
\end{Verbatim}
\end{figure}

The data included in the example file is a truncated version of the G clone 
DNA methylation alignment with the first 116 sites. Setting  \texttt{datatype=dimethyl}
in the data block automatically applies the dimethyl substitution model.

\subsection{Specifying the Model}

The model is specified in the following mrbayes block:

\begin{figure}[h]
\centering
\begin{BVerbatim}
begin mrbayes;
prset dimethylrate=dirichlet(0.6, 0.4);
prset readerrpr=fixed(0.001);
mcmc ngen=10000 samplefreq=500 burnin=1000;
end;
\end{BVerbatim}
\end{figure}

The \texttt{dimethylrate} option in the \texttt{prset} command above sets the prior distribution
for the DNA methylationirate paramters \(a\) and \(b\) to a \(Dirichlet(0.6,0.4)\) distribution. 
The DNA methylation rates can also be fixed if desired, for example the command 
\texttt{prset dimethylrate=fixed(0.6, 0.4);} will set the prior distribution of \(a\) to be a point mass
at \(0.6\). If no prior is specified, the rate prior is set to \(Dirichlet(0.5, 0.5)\).
If desired, the DNA methylation rates may be fixed; for example \texttt{prset dimethylrate=fixed(0.6,0.4)}.
If the DNA methylation rates are provided such that \(a + b \neq 1\), the rates will be normalized by 
dividing by \(a + b\).

The read error rate is fixed at \(0.001\) with the option 
\texttt{readerrpr=fixed(0.001);}, a uniform distribution can be specified using (for example)
\texttt{readerrpr=uniform(0.0, 0.001);}, to allow the read error rate to vary between 0 and 0.001.
If no prior is specified, the read error rate is set to 0.

\end{document}
